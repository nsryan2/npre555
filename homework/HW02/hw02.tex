% use the answers clause to get answers to print; otherwise leave it out.
\documentclass[12pts,answers]{exam}
%\documentclass[12pts]{exam}
\RequirePackage{amssymb, amsfonts, amsmath, latexsym, verbatim, xspace, setspace}
\usepackage{graphicx}

% By default LaTeX uses large margins.  This doesn't work well on exams; problems
% end up in the "middle" of the page, reducing the amount of space for students
% to work on them.
\usepackage[margin=1in]{geometry}
\usepackage{enumerate}
\usepackage[hidelinks]{hyperref}

% Here's where you edit the Class, Exam, Date, etc.
\newcommand{\class}{NPRE 555}
\newcommand{\term}{Fall 2023}
\newcommand{\assignment}{HW 2}
\newcommand{\duedate}{2023.09.20}
%\newcommand{\timelimit}{50 Minutes}

\newcommand{\nth}{n\ensuremath{^{\text{th}}} }
\newcommand{\ve}[1]{\ensuremath{\mathbf{#1}}}
\newcommand{\Macro}{\ensuremath{\Sigma}}
\newcommand{\vOmega}{\ensuremath{\hat{\Omega}}}

% For an exam, single spacing is most appropriate
\singlespacing
% \onehalfspacing
% \doublespacing

% For an exam, we generally want to turn off paragraph indentation
\parindent 0ex

%\unframedsolutions

\begin{document}

% These commands set up the running header on the top of the exam pages
\pagestyle{head}
\firstpageheader{}{}{}
\runningheader{\class}{\assignment\ - Page \thepage\ of \numpages}{Due \duedate}
\runningheadrule

\class \hfill \term \\
\assignment \hfill Due \duedate\\
\rule[1ex]{\textwidth}{.1pt}
%\hrulefill

%%%%%%%%%%%%%%%%%%%%%%%%%%%%%%%%%%%%%%%%%%%%%%%%%%%%%%%%%%%%%%%%%%%%%%%%%%%%%%%%%%%%%
%%%%%%%%%%%%%%%%%%%%%%%%%%%%%%%%%%%%%%%%%%%%%%%%%%%%%%%%%%%%%%%%%%%%%%%%%%%%%%%%%%%%%
\begin{itemize}
        \item Show your work.
        \item This work must be submitted online as a \texttt{.pdf} through gradescope.
        \item If this work is completed with the aid of a numerical program
                (such as Python, Wolfram Alpha, or MATLAB) all scripts and data
                must be submitted in addition to the \texttt{.pdf}.
        \item If you work with anyone else, document what you worked on together.
\end{itemize}
\rule[1ex]{\textwidth}{.1pt}

% ---------------------------------------------
\begin{questions}
        \question[10] We have defined the angular neutron density n($\vec{r}$, $\hat{\Omega}$, E, t) in terms of the nuetron energy
        E and the direction of motion $\hat{\Omega}$, but one could define an angular density that depends instead on the
        neutron velocity $\vec{v}$, n($\vec{r}$, $\vec{v}$, t). Calculate the relationship between these two dependent variables.


        \begin{solution}
            $n(\vec{r},\vec{\Omega}, E, t) = \left(\frac{v}{m}\right) n(\vec{r},\vec{v},t)$
        \end{solution}
        \pagebreak
        \question Prove the relations we mentioned in class:

        % ---------------------------------------------
        \begin{parts}
                \part[5] $n(\vec{r},\vec{\Omega}, E, t) = \left(\frac{1}{mv}\right) n(\vec{r},v, \vec{\Omega},t)$
                \part[5] $n(\vec{r}, v, \vec{\Omega}, t) = v^2  n(\vec{r},\vec{v}, t)$

        \end{parts}
        \pagebreak

        % ---------------------------------------------
        \question Two thermal neutron beams are injected from opposite directions into a thin sample of $^{235}U$. At a given
        point in the sample, the beam intensities are $10^{12}$ neutrons/cm$^2$-s from the left and 2 * $10^{12}$ neutrons/cm$^2$-s from the right. Compute:

        \begin{parts}
            \part[10] the neutron flux and current density at this point
            \part[10] the fission reaction rate density at this point
        \end{parts}
        \pagebreak

        % ---------------------------------------------
        \question Suppose that the angular neutron density is given by
        \begin{equation*}
            n(\vec{r},\hat{\Omega}) = \frac{n_0}{4\pi}(1-\cos(\theta))
        \end{equation*}
        where $\theta$ is the angle between $\hat{\Omega}$ and the z-axis. If A is the area perpendicular to the z-axis, then what is the number of neutrons passing through the area A per second:

        \begin{parts}
            \part[5] er unit solid angle at an angle of 45 degrees with the z-axis,
            \part[5] from the negative z to the positive z direction,
            \part[5] net
            \part[5] total
        \end{parts}
        \pagebreak

        % ---------------------------------------------
        \question[40] A purely absorbing half-plane medium, in which $\sigma = 1$,
        contains a sourse emitting $1\left[\frac{n}{cm^3s}\right]$. Determine
                the intensity and angular distribution of the flux and the
                current at the surface. (Bell and Glasstone 1.3)
        \pagebreak

        % ---------------------------------------------
        \question[45] Suppose there is a purely absorbing region of finite
        thickness. It is desired to represent this region as an absorbing
        region accross which the neutron angular density is discontinuous.
        Derive the discontinuity which is required in the angular density.
        (Bell and Glasstone 1.10)

\end{questions}



%\bibliographystyle{plain}
%\bibliography{hw01}
\end{document}
